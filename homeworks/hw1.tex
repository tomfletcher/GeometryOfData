\documentclass[11pt]{article}
\usepackage{fullpage,url,hyperref}
\usepackage{amsmath}
\usepackage{amssymb}

\usepackage[letterpaper,top=1in,bottom=1in,left=1in,right=1in,nohead]{geometry}

\setlength{\parindent}{0in}
\setlength{\parskip}{6pt}

\begin{document}
\thispagestyle{empty}
{\large{\bf ECE / CS 6501: Geometry of Data \hfill Due Tu 9/17}}\\

{\LARGE{\bf Homework 1: Getting Started }}
\vspace{0.2\baselineskip}
\hrule

{\bf Instructions:} Submit a single Jupyter notebook (.ipynb) of your work to
Canvas by 11:59pm on the due date. All code should be written in Python. {\bf Be
  sure to show all the work involved in deriving your answers! If you just give
  a final answer without explanation, you may not receive credit for that
  question.}

You may discuss the concepts with your classmates, but write up the answers
entirely on your own. Do not look at another student's answers, do not copy
answers from the internet or other sources, and do not show your answers to
anyone. {\bf Remember to cite any sources you use, including any prompts given
  to AI chatbots.}

\begin{enumerate}

\item Consider the set of all {\bf closed} intervals in the real line with
  non-zero length:
  $$\mathcal{B} = \{[a, b] : a < b\}.$$
  Is this a valid basis for a topology on $\mathbb{R}$? Why or why not?

\item An {\bf open map} is a mapping between two topological spaces,
  $f : X \rightarrow Y$, such that for any open set $U \subseteq X$, the image
  $f(U)$ is an open set of $Y$.

  \begin{enumerate}
  \item Let $X = \{a, b, c\}$ and $Y = \{a, b, c\}$ both represent the three
    point set example from lecture but with potentially different topologies,
    $\mathcal{T}_X$ and $\mathcal{T}_Y$, respectively. Construct an example of
    topologies $\mathcal{T}_X$ and $\mathcal{T}_Y$ along with a mapping
    $f: X \rightarrow Y$ that is continuous but not open.

  \item With the same sets $X$ and $Y$ as in part (a), construct topologies and
    a mapping ${f:X \rightarrow Y}$ that is open but not continuous.

  \item Give an example of a real-valued function
    $f : \mathbb{R} \rightarrow \mathbb{R}$ that is continuous but not open.

  \item Give an example of a real-valued function
    $f : \mathbb{R} \rightarrow \mathbb{R}$ that is open but not
    continuous. {\bf Hint:} You'll find such a function is quite difficult to
    come up with. You are certainly welcome to use a web search and/or AI to
    answer this question. Make sure to verify the example is correct and cite
    your sources!
  \end{enumerate}

\item Write a Python function to compute the stereographic projection,
  $f: S^2 \rightarrow \mathbb{R}^2$, from the north pole $(0,0,1)$ of the
  2-sphere to the $z = 0$ plane. You may want to see the Wikipedia page for the
  equation:
  \begin{center}
    \url{https://en.wikipedia.org/wiki/Stereographic_projection}
  \end{center}

  \begin{enumerate}
  \item What is the image of a longitudinal line, i.e., a great circle passing
    through the north and south poles? Use your function to plot several images
    of logitudinal lines in the plane.

  \item What is the image of a latitude line, i.e., a circle formed by
    intersection with a horizontal plane ($z = \mathrm{const.}$)? Plot several
    images of latitude lines in the plane.

  \item What is the image of an arbitrary great circle (one that is {\bf not} a
    longitudinal line or the equator)? Again, plot several in the plane.
  \end{enumerate}

\item Write a Python function to sample uniformly random points on the
  $d$-sphere, $S^d$. Do this by first sampling $d+1$ i.i.d. standard Gaussian
  random variables, $N(0,1)$, to get a point
  $x = (x_1, x_2, \ldots, x_{d+1}) \in \mathbb{R}^{d+1}$. Then project this
  point to the unit sphere by $x \mapsto \frac{x}{\|x\|}$.

  Repeat the following experiments for $d = 2, 4, 10, 100, 1000$
  \begin{enumerate}
  \item Sample 10,000 points on $S^d$ and compute their distance from the
    ``equator'', or great circle with $x_1 = 0$. Distance to the equator should
    be arc distance along the sphere. For each value of $d$, plot a histogram of
    these distances. What do you notice as $d$ increases? What does this say
    about uniform random points on the sphere? Would this result be the same or
    different if you used a different great circle (possibly not axis-aligned)?

  \item Sample 1,000 pairs of points on $S^d$ and compute the angle between
    them. Again, for each value of $d$, plot a histogram of these pairwise
    angles. What do you notice as $d$ increases and what does it say about the
    geometry of uniformly random points?
  \end{enumerate}
  
\end{enumerate}
\end{document}
